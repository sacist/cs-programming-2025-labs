\documentclass[]{vvsu}

\vvsuyear{2025}

%%%%%%%%%%%%%%%%%%%

\usepackage{graphicx} % для изображений
\usepackage{tabularray} % для таблиц
\usepackage{siunitx} % для обозначений (процент, градус)
\usepackage{listings} % для листингов кода

% Список путей, где будут искаться изображения и файлы
\graphicspath{{images/}}

% Файл со списком источников (не используется)
% \addbibresource{./references.bib}

% Автор документа
\author{А.Р. Рублев}

% Настройка стилей для листингов кода
\usepackage{fontspec}
\setmonofont{Liberation Mono}  % вместо Consolas

\usepackage{xcolor}
\usepackage{listings}

% ======================
% YAML язык
% ======================
\makeatletter
\newcommand\language@yaml{yaml}
\expandafter\expandafter\expandafter\lstdefinelanguage
\expandafter{\language@yaml}
{
  keywords={true,false,null,y,n},
  keywordstyle=\color{darkgray},
  basicstyle=\setmonofont{Liberation Mono}\fontsize{8}{8}\linespread{1}\selectfont,
  sensitive=false,
  comment=[l]{\#},
  morecomment=[s]{/*}{*/},
  commentstyle=\color{purple},
  stringstyle=\color{blue},
  moredelim=[l][\color{orange}]{\&},
  moredelim=[l][\color{magenta}]{*},
  moredelim=**[il][\color{red}{:}\color{blue}]{:},
  morestring=[b]',
  morestring=[b]",
  literate =    {---}{{\ProcessThreeDashes}}3
                {>}{{\textcolor{red}\textgreater}}1
                {|}{{\textcolor{red}\textbar}}1
                {\ -\ }{{\ -\ }}3,
}
\lst@AddToHook{EveryLine}{\ifx\lst@language\language@yaml\color{black}\fi}
\makeatother

% ======================
% JSON язык
% ======================
\lstdefinelanguage{json}{
    basicstyle=\fontsize{8}{8}\linespread{1}\selectfont\ttfamily,
    sensitive=false,
    stringstyle=\color{blue},
    string=[s]{":\ "}{"},
    literate=
        *{0}{{{\color{red}0}}}{1}
         {1}{{{\color{red}1}}}{1}
         {2}{{{\color{red}2}}}{1}
         {3}{{{\color{red}3}}}{1}
         {4}{{{\color{red}4}}}{1}
         {5}{{{\color{red}5}}}{1}
         {6}{{{\color{red}6}}}{1}
         {7}{{{\color{red}7}}}{1}
         {8}{{{\color{red}8}}}{1}
         {9}{{{\color{red}9}}}{1}
}

% ======================
% JavaScript язык
% ======================
\lstdefinelanguage{JavaScript}{
    keywords={break,case,catch,continue,debugger,default,delete,do,else,false,
      finally,for,function,if,in,instanceof,new,null,return,switch,this,throw,true,
      try,typeof,var,let,const,while,with,await,async},
    keywordstyle=\color{blue}\bfseries,
    ndkeywords={class,export,import},
    ndkeywordstyle=\color{darkgray}\bfseries,
    identifierstyle=\color{black},
    sensitive=true,
    comment=[l]{//},
    morecomment=[s]{/*}{*/},
    commentstyle=\color{purple}\ttfamily,
    stringstyle=\color{orange}\ttfamily,
    morestring=[b]',
    morestring=[b]",
    basicstyle=\ttfamily\small,
    breaklines=true,
    showstringspaces=false,
}

% ======================
% Общий стиль для листингов
% ======================
\definecolor{codegray}{rgb}{0.5,0.5,0.5}
\definecolor{backcolour}{rgb}{0.95,0.95,0.92}
\lstdefinestyle{codestylelst}{
    backgroundcolor=\color{backcolour},
    numberstyle=\color{codegray}\ttfamily,
    breakatwhitespace=false,
    breaklines=true,
    captionpos=b,
    keepspaces=true,
    numbers=left,
    numbersep=5pt,
    showspaces=false,
    showstringspaces=false,
    showtabs=false,
    tabsize=2
}

% Применяем стиль ко всем листингам по умолчанию
\lstset{style=codestylelst}


%%%%%%%%%%%%%%%%%%%

\begin{document}

% Шапка
\vvsuhead{\linespread{1}\selectfont{}МИНОБРНАУКИ РОССИИ\\
\vspace{10pt}Федеральное государственное бюджетное образовательное учреждение\\
высшего образования\\
\fontsize{13}{13}\selectfont{}<<ВЛАДИВОСТОКСКИЙ ГОСУДАРСТВЕННЫЙ УНИВЕРСИТЕТ>>\\
(ФГБОУ ВО <<ВВГУ>>)\\
\vspace{10pt}\fontsize{12}{12}\selectfont{}ИНСТИТУТ ИНФОРМАЦИОННЫХ ТЕХНОЛОГИЙ И АНАЛИЗА ДАННЫХ\\
КАФЕДРА ИНФОРМАЦИОННЫХ ТЕХНОЛОГИЙ И СИСТЕМ}

% Название отчета
\title{Отчет\\по лабораторной работе №4}
\subtitle{по дисциплине\\<<Информатика и программирование>>}

% Участники работы
\member{Студент\\ гр. БИС-25-3}{А.Р. Рублев}
\member{Ассистент\\ преподавателя}{М.В. Водяницкий}

% Вывод титульника
\maketitle

% Задание
% Задание
\begin{addition}{Задание}
  Выполнить задания на Python и оформить отчет по стандартам ВВГУ.

  \textit{\textbf{Задание 1.}}  
  Написать программу, которая определяет, как будет вести себя кондиционер. Если температура в помещении 20 градусов и выше, то кондиционер выключается, если меньше - включается. Температура должна вводится пользователем с консоли.

  \textit{\textbf{Задание 2.}}  
  Год делится на четыре сезона: зима, весна, лето и осень. Написать программу, которая запрашивает у пользователя номер месяца и выводит к какому сезону этот месяц относится.

  \textit{\textbf{Задание 3.}}  
  Считается, что один год, прожитый собакой, эквивалентен семи человеческим годам. При этом зачастую не учитывается, что собаки становятся абсолютно взрослыми уже к двум годам. Таким образом, многие предпочитают каждый из первых двух лет жизни собаки приравнивать к 10.5 годам человеческой жизни, а все последующие к 4.

  Написать программу, которая будет переводить собачий возраст в человеческий. Программа должна корректно обрабатывать входные данные и выводить соответствующие сообщения об ошибках:
  \begin{vvsu_list}
  \item Если вводится не число
  \item Если вводится число меньше 1
  \item Если вводится число большее 22
  \end{vvsu_list}

  \textit{\textbf{Задание 4.}}  
  Число делиться на 6 только в случае соблюдения двух условий:
  \begin{vvsu_list}
    \item Последняя цифра четная
    \item Сумма всех цифр делиться на 3
  \end{vvsu_list}

  Написать программу, которая выведет делиться ли введенное число на 6 или нет.

  \textit{\textbf{Задание 5.}}
  Написать программу, которая будет проверять пароль на надежность. Пароль считается надежным, если его длина не менее 8 символов и если он содержит:
  \begin{vvsu_list}
    \item Заглавные буквы латиницы
    \item Строчные буквы латиницы
    \item Числа
    \item Специальные знаки
  \end{vvsu_list}

  В случае, если пароль не проходит по одному из условий, необходимо сообщить пользователю каким именно условиям он не удовлетворяет.

  \textit{\textbf{Задание 6.}}  
  Написать программу, которая определяет, является ли введенный пользователем год високосным. Год считается високосным, если он делится на 4, но не делится на 100, либо если он делится на 400.

  \textit{\textbf{Задание 7.}}  
  Написать программу, которая запрашивает у пользователя три числа и выводит на экран наименьшее из них. 
  При решении нельзя использовать встроенные функции min() и max().

  \textit{\textbf{Задание 8.}}  
  В магазине проводится акция. Акция работает по следующим правилам:
  \begin{vvsu_list}
    \item до 1000	- 0%
    \item 1000–5000	- 5%
    \item 5000–10000 - 10%
    \item более 10000	- 15%
  \end{vvsu_list}
  \textit{\textbf{Задание 9.}}  
  Написать программу, которая определяет время суток по введенному часу (целое число от 0 до 23).
  \begin{vvsu_list}
    \item 0–5	- Ночь
    \item 6–11 - Утро
    \item 12–17	- День
    \item 18–23	- Вечер
  \end{vvsu_list}
  
  \textit{\textbf{Задание 10.}}  
  Написать программу, которая определяет, является ли введенное число простым. Число называется простым, если оно больше 1 и делится только на 1 и само себя. Программа должна корректно обрабатывать некорректный ввод и выводить соответствующие сообщения об ошибках.
\end{addition}

% Содержание
\toc

% Глава - Выполнение работы
\section{Выполнение работы}

% Подглава - Задание 1
\subsection{Задание 1}

Написать программу, которая определяет, как будет вести себя кондиционер. Если температура в помещении 20 градусов и выше, то кондиционер выключается, если меньше - включается. Температура должна вводится пользователем с консоли.

\begin{vvsu_figure}{Листинг программы для задания 1}{fig:code_task_1}
  \begin{minipage}{.75\textwidth}
    \lstinputlisting[language=Python,basicstyle=\fontsize{10}{10}\linespread{1}\selectfont\ttfamily]{code/task1.js}
  \end{minipage}
\end{vvsu_figure}


% Подглава - Задание 2
\subsection{Задание 2}

Год делится на четыре сезона: зима, весна, лето и осень. Написать программу, которая запрашивает у пользователя номер месяца и выводит к какому сезону этот месяц относится.

\begin{vvsu_figure}{Листинг программы для задания 2}{fig:code_task_2}
  \begin{minipage}{.75\textwidth}
    \lstinputlisting[language=Python,basicstyle=\fontsize{10}{10}\linespread{1}\selectfont\ttfamily]{code/task2.js}
  \end{minipage}
\end{vvsu_figure}


% После выполнения программы в консоли отображается имя и возраст пользователя.

% Подглава - Задание 3ыы
\subsection{Задание 3}

Считается, что один год, прожитый собакой, эквивалентен семи человеческим годам. При этом зачастую не учитывается, что собаки становятся абсолютно взрослыми уже к двум годам. Таким образом, многие предпочитают каждый из первых двух лет жизни собаки приравнивать к 10.5 годам человеческой жизни, а все последующие к 4.

Написать программу, которая будет переводить собачий возраст в человеческий. Программа должна корректно обрабатывать входные данные и выводить соответствующие сообщения об ошибках:

\begin{vvsu_figure}{Листинг программы для задания 3}{fig:code_task_3}
  \begin{minipage}{.75\textwidth}
    \lstinputlisting[language=Python,basicstyle=\fontsize{10}{10}\linespread{1}\selectfont\ttfamily]{code/task3.js}
  \end{minipage}
\end{vvsu_figure}


% Подглава - Задание 4
\subsection{Задание 4}

Написать программу, которая выведет делиться ли введенное число на 6 или нет.

\begin{vvsu_figure}{Листинг программы для задания 4}{fig:code_task_4}
  \begin{minipage}{.75\textwidth}
    \lstinputlisting[language=Python,basicstyle=\fontsize{10}{10}\linespread{1}\selectfont\ttfamily]{code/task4.js}
  \end{minipage}
\end{vvsu_figure}

% Подглава - Задание 5
\subsection{Задание 5}

Написать программу, которая будет проверять пароль на надежность. Пароль считается надежным, если его длина не менее 8 символов и если он содержит:

  Заглавные буквы латиницы
  Строчные буквы латиницы
  Числа
  Специальные знаки
В случае, если пароль не проходит по одному из условий, необходимо сообщить пользователю каким именно условиям он не удовлетворяет.

\begin{vvsu_figure}{Листинг программы для задания 5}{fig:code_task_5}
  \begin{minipage}{.75\textwidth}
    \lstinputlisting[language=Python,basicstyle=\fontsize{10}{10}\linespread{1}\selectfont\ttfamily]{code/task5.js}
  \end{minipage}
\end{vvsu_figure}

% Подглава - Задание 6
\subsection{Задание 6}

Написать программу, которая определяет, является ли введенный пользователем год високосным. Год считается високосным, если он делится на 4, но не делится на 100, либо если он делится на 400.



\begin{vvsu_figure}{Листинг программы для задания 6}{fig:code_task_6}
  \begin{minipage}{.75\textwidth}
    \lstinputlisting[language=Python,basicstyle=\fontsize{10}{10}\linespread{1}\selectfont\ttfamily]{code/task6.js}
  \end{minipage}
\end{vvsu_figure}

% Подглава - Задание 7
\subsection{Задание 7}

Написать программу, которая запрашивает у пользователя три числа и выводит на экран наименьшее из них. При решении нельзя использовать встроенные функции min() и max().



\begin{vvsu_figure}{Листинг программы для задания 7}{fig:code_task_7}
  \begin{minipage}{.75\textwidth}
    \lstinputlisting[language=Python,basicstyle=\fontsize{10}{10}\linespread{1}\selectfont\ttfamily]{code/task7.js}
  \end{minipage}
\end{vvsu_figure}

% Подглава - Задание 8
\subsection{Задание 8}

В магазине проводится акция. Акция работает по следующим правилам:



\begin{vvsu_figure}{Листинг программы для задания 8}{fig:code_task_8}
  \begin{minipage}{.75\textwidth}
    \lstinputlisting[language=Python,basicstyle=\fontsize{10}{10}\linespread{1}\selectfont\ttfamily]{code/task8.js}
  \end{minipage}
\end{vvsu_figure}

% Подглава - Задание 9
\subsection{Задание 9}

Написать программу, которая определяет время суток по введенному часу (целое число от 0 до 23).

\begin{vvsu_figure}{Листинг программы для задания 9}{fig:code_task_9}
  \begin{minipage}{.75\textwidth}
    \lstinputlisting[language=Python,basicstyle=\fontsize{10}{10}\linespread{1}\selectfont\ttfamily]{code/task9.js}
  \end{minipage}
\end{vvsu_figure}

% Подглава - Задание 10
\subsection{Задание 10}

Написать программу, которая определяет, является ли введенное число простым. Число называется простым, если оно больше 1 и делится только на 1 и само себя. Программа должна корректно обрабатывать некорректный ввод и выводить соответствующие сообщения об ошибках.

\begin{vvsu_figure}{Листинг программы для задания 10}{fig:code_task_10}
  \begin{minipage}{.75\textwidth}
    \lstinputlisting[language=Python,basicstyle=\fontsize{10}{10}\linespread{1}\selectfont\ttfamily]{code/task10.js}
  \end{minipage}
\end{vvsu_figure}


% \begin{application}{Приложение А}{Приложение А\\ \vspace{1em} Описание процесса работы с драйверами}
%   Самое обычное тестовое приложение
% \end{application}

\end{document}